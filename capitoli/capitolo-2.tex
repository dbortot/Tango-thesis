% !TEX encoding = UTF-8
% !TEX TS-program = pdflatex
% !TEX root = ../tesi.tex
% !TEX spellcheck = it-IT

%**************************************************************
\chapter{Processi di sviluppo}
\label{cap:processi-metodologie}
%**************************************************************

\intro{Una buona definizione dei processi produttivi è fondamentale per il successo del progetto}\\

%**************************************************************
\section{Processo sviluppo prodotto}
In ambito aziendale si è scelto di provare, per questo progetto, un processo di sviluppo \emph{software} basato sulla filosofia \emph{Lean}.\\
Dato che l'obiettivo principale era fornire un prototipo ci si è limitati solamente alle tre fasi iniziali dello sviluppo di \emph{Lean}, ovvero \emph{Kick-Off}, \emph{Concept Preview} e \emph{Product Prototype}.


\subsection{Kick-Off}
Questa è la prima fase dello sviluppo \emph{software}, coincide con la prima riunione ufficiale del team di progetto, aperta anche agli \emph{Stakeholder}.\\
Si pone lo scopo di iniziare la fase l'\emph{allestimento} e l'\emph{avviamento} in cui viene determinata la natura e lo scopo del progetto.\\

\subsection{Concept Preview}
Fase in cui è reso disponibile un primo campione di prova del prodotto, detto \emph{concept}. Esso può essere incompleto e affetto da errori, ma deve essere in grado di dimostrare agli \emph{stakeholder} le caratteristiche principali che avrà il prodotto finito.\\
Esso è soggetto a un riesame che ha lo scopo di valutare se è in linea con gli obiettivi definiti nella \emph{Value Proposition}, la \emph{milestone} non può essere raggiunta senza che questo riesame abbia esito positivo.\\
Questa fase coincide anche con l'inizio della progettazione, che deve essere portata avanti fino ad un livello di dettaglio ritenuto opportuno dal \emph{team}.\\

\subsection{Product Prototype}
Fase in cui è messo a disposizione il primo prototipo del nuovo prodotto, completo nelle sue funzioni (sviluppo finito) ma non ancora messo a punto mediante verifiche e correzioni, per garantirne funzionalità e prestazioni. Il prototipo deve essere ad un stato tale da poter essere dato in valutazione agli \emph{stakeholder}.\\
Esso è soggetto a un riesame che ha lo scopo di valutare se è in linea con gli obiettivi definiti sia \emph{Value Proposition} che nella \emph{Requirements Specification}, la \emph{milestone} non può essere raggiunta senza che questo riesame abbia esito positivo.\\
Questa fase coincide con il termine della fase di progettazione e l'inizio della fase di esecuzione, cioè l'insieme dei processi necessari a soddisfare i requisiti del progetto.

\subsection{Fasi successive}
Le fasi successive, ovvero \emph{Product Design Freeze} e \emph{Start Of Production}, possono essere avviate nel futuro a partire dal \emph{Product Prototype} se ciò verrà ritenuto opportuno dall'azienda.