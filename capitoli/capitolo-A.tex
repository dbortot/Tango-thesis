% !TEX encoding = UTF-8
% !TEX TS-program = pdflatex
% !TEX root = ../tesi.tex
% !TEX spellcheck = it-IT

%**************************************************************
\chapter{Appendice A}
%**************************************************************
\chapter{Glossario}\label{appendix:Glossario}

\subsubsection{API}
Con \emph{API} o \emph{Application Programming Interface} si intende ogni insieme di metodi e procedure resi disponibili al programmatore. Di solito raggruppati in a formare un set di strumenti specifici per l'espletamento di un determinato compito.

\subsubsection{Artefatto}
Gli \emph{artefatti} sono degli elementi presenti all'interno di una ricostruzione \emph{Point Cloud} ma non nella realtà, dovuti errori nella rilevazione. Sono generalmente di forma planare e sospesi qualche centimetro al di sopra del pavimento.

\subsubsection{Design Pattern}
Un \emph{Design Pattern} è una soluzione progettuale ad un problema ricorrente in un certo contesto.

\subsubsection{Ghosting}
Il problema del \emph{Ghosting} affligge alcune registrazioni effettuate con il prodotto: è dovuto ad una errata stima della posizione del dispositivo e produce ricostruzioni tridimensionali affette da errori. Se visualizzate graficamente il tali ricostruzioni presentano l'oggetto come se fosse sdoppiato, ad esempio come in figura \ref{fig:no_drift_correction}.

\subsubsection{Feature}
Una \emph{feature} è un punto particolare nello spazio che il processo di \emph{Area Learning} ritiene facilmente riconoscibile, e che quindi è salvato ed usato come punto di riferimento in un determinato ambiente. Un'area ha generalmente qualche centinaio di \emph{feature}.

\subsubsection{Fotocamera Fish-eye}
Una Fotocamera \emph{Fish-eye} è un obbiettivo grandangolare che abbraccia un angolo di campo di circa 180 gradi, in particolare quello in dotazione nel dispositivo usato è in bianco e nero.

\subsubsection{Mesh}
Una \emph{mesh} o \emph{mesh} poligonale è una collezione di vertici, spigoli e facce che definiscono la forma di un oggetto tridimensionale.

\subsubsection{Motion Tracking}
Il \emph{Motion Tracking} è una tecnica che permette al dispositivo di tracciare i suoi movimenti all'interno di un area.

\subsubsection{Point Cloud}
Un \emph{Point Cloud} è modello matematico che descrive un oggetto tridimensionale semplicemente come insieme del maggior numero possibile di punti dell'oggetto stesso.\\
È molto usato in ambito di \emph{Computer vision} e realtà aumentata.

\subsubsection{Stakeholder}
Gli \emph{Stakeholder}, o portatori di interesse, sono l'insieme delle persone che a vario titolo sono coinvolte nel ciclo di vita del \emph{Software} con influenza sul prodotto.

\subsubsection{Project Tango}
Il progetto \emph{Tango} è un progetto promosso e portato avanti da \emph{Google}. Si propone di promuovere nuovi tipi di dispositivi dotati di \emph{Hardware} innovativo, come fotocamera \emph{Fish-Eye}, sensore di profondità etc. Grazie ai loro sensori i dispositivi \emph{Tango} sono in grado di avere una certa conoscenza dell'ambiente che li circonda: sono in grado di tracciare il proprio movimento ed orientazione, di ricordare gli ambienti dove sono già stati, di avere una visione tridimensionale degli oggetti.

\subsubsection{UML - Unified Modelling Language}
Famiglia di notazioni grafiche che si basano su un singolo meta-modello e servono a
supportare la descrizione e il progetto dei sistemi software.

\subsubsection{Voxel}
Un \emph{voxel}, detto anche \emph{pixel} volumetrico, è un elemento di volume che rappresenta un valore di intensità di segnale o di colore in uno spazio tridimensionale.





