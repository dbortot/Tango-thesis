% !TEX encoding = UTF-8
% !TEX TS-program = pdflatex
% !TEX root = ../tesi.tex
% !TEX spellcheck = it-IT

%**************************************************************
\chapter{Studio di fattibilità ed Analisi dei rischi}

\intro{Il progetto si è subito presentato come sperimentale, impegnativo e facilmente soggetto a fallimento. Per questo si è reso necessario studiarne attentamente la fattibilità e valutarne i rischi}\\

%**************************************************************
\section{Introduzione al progetto}

Data la natura innovativa del progetto è stato necessario produrre diversi prototipi ed
effettuare l’Analisi dei Rischi e lo Studio di Fattibilità in diverse fasi.
Seguendo quest'approccio è stato possibile valutare le potenzialità e i limiti del \emph{device Tango} e delle API annesse, e considerare l'applicabilità degli algoritmi general purpose della libreria \gls{pcl} al caso particolare del progetto in questione.

%**************************************************************
\section{Studio di fattibilità}
Prima di iniziare il progetto è stato effettuato un accurato studio di fattibilità basato sullo studio della libreria \emph{PCL} e della vasta quantità di esempi d'utilizzo della stessa reperibili online. Si sono cercate poi soluzioni per il calcolo del volume di una mesh.\\
Per il lato \emph{client} si sono esplorate soluzioni per la visualizzazione di mesh salvate in formato OBJ e per la comunicazione \emph{client}-\emph{server}.

\subsection{Elaborazione dei Point Cloud}
\'E parso subito evidente che l'applicativo necessita di un formato standard, trasferibile via Web, per il trattamento di Point Cloud, e delle funzionalità di I/O associate. Inoltre è stato necessario esplorare le funzionalità disponibli per il filtraggio ed il meshing dei Point Cloud acquisiti. Per ultimo si è cercato un affidabile metodo matamatico per il calcolo del volume di una mesh. A tal riguardo è stata studiata estensivamente la documentazione \emph{PCL}, che offre anche numerosi esempi d'utilizzo:
\begin{itemize}
\item\textbf{The PCD (Point Cloud Data) file format} \cite{site:PCD}
\item\textbf{Reading Point Cloud data from PCD files} \cite{site:PCD-read}
\item\textbf{Writing Point Cloud data to PCD files} \cite{site:PCD-write}
\item\textbf{Filtering a PointCloud using a PassThrough filter} \cite{site:passthrough}
\item\textbf{Downsampling a PointCloud using a VoxelGrid filter} \cite{site:voxel}
\item\textbf{Removing outliers using a StatisticalOutlierRemoval filter} \cite{site:outliers}
\item\textbf{Plane model segmentation}\cite{site:segmentation}
\item\textbf{Euclidean Cluster Extraction}\cite{site:cluster}
\item \textbf{Fast triangulation of unordered point clouds}\cite{site:meshing}
\item \textbf{Fitting trimmed B-splines to unordered point clouds}\cite{site:bsplines}
\end{itemize}
\noindent
Per il calcolo del volume invece, punto focale del progetto, si è studiato quanto descritto in \cite{paper:volume}

\subsection{Visualizzazione di file in formato OBJ}
Per visualizzare le mesh prodotte dal \emph{server} su \emph{tablet} è stato necessario ricercare esempi di applicativi che permettessero di trattare oggetti tridimensionali in formato OBJ.\\
A tal riguardo si sono studiate due applicazioni Android:
\begin{itemize}
\item\textbf{3D viewer}\cite{site:3d-viewer}
\item\textbf{3D Model Viewer Open Source}\cite{site:3d-viewer-os}
\end{itemize}
\noindent
In particolare la seconda applicazione essendo open-source (realizzata da Andres Oviedo, il codice è liberamente reperibile dalla repository\footcite{https://github.com/andresoviedo/android-3D-model-viewer} GitHub) si è rivelata una valida e disponibile implementazione della funzionalità richiesta.

\subsection{Comunicazione client-server}
Per implementare la comunicazione \emph{client}-\emph{server} la scelta più logica è stata di esplorare la documentazione ufficiale Android, che mette a disposizione il package \textbf{Apache HTTP}\cite{site:apache}, che è stato però deprecato dalla versione 6.0 di Android, quindi si è studiato come alternativa l'ottima e leggera libreria open source \textbf{OkHttp}\cite{site:okhttp} che risponde alle necessità richieste.

\subsection{Conclusioni}
Alla luce della disponibilità e qualità degli esempi sopracitati il progetto è apparso fattibile. La documentazione e le applicazioni studiate infatti hanno dato prova delle potenzialità della libreria \emph{PCL} e della possibilità concreta di calcolare il volume esatto di una mesh tridimensionale. \'E quindi stato possibile dare il via al ciclo di vita del progetto software.


\newpage
%**************************************************************
\section{Analisi preventiva dei rischi}

Durante la fase di analisi iniziale sono stati individuati alcuni possibili rischi a cui si potrà andare incontro.
Si è quindi proceduto a elaborare delle possibili soluzioni per far fronte a tali rischi.\\

\subsection{Rischi generali}

\begin{risk}{Limiti fisici del tablet}
	\riskprob{Alta}
	\riskimpact{Medio}
    \riskdescription{Il dispositivo è dotato di sensori IR che sfruttano la riflessione della luce per determinare la distanza dei punti che è in grado di individuare, e di fotocamere RGB. Superfici lucide/riflettenti o tendenti al nero possono compromettere gravemente la qualità della misurazione, così come gli ambienti con illuminazione scarsa o troppo intensa}
    \risksolution{ Attento studio delle potenzialità dell'hardware del \emph{tablet}\footcite{https://developers.google.com/tango/hardware/tablet} e del loro utilizzo\footcite{https://developers.google.com/tango/overview/depth-perception} nelle Tango API}
    \label{risk:device-limits} 
\end{risk}

\begin{risk}{Incomprensibilità dello stato del progetto}
	\riskprob{Media}
	\riskimpact{Medio}
    \riskdescription{Venendo inserito in un progetto già avviato, lo studente necessita di comprendere a fondo lo stato dell'arte dell'applicativo}
    \risksolution{Lo studente deve, consultando la documentazione creata e collaborando strettamente con gli altri elementi del team e con il tutor aziendale, assicurasi di comprendere quanto è stato fatto, come è stato fatto e le necessità future del progetto.}
    \label{risk:project-status} 
\end{risk}

\begin{risk}{Scarsa competenza nello sviluppo Android}
	\riskprob{Alta}
	\riskimpact{Basso}
    \riskdescription{Lo studente ha scarse conoscenze dello sviluppo Android, ma conosce il linguaggio Java}
    \risksolution{Lo studente deve effettuare un adeguato periodo di studio ed assumere familiarità con la documentazione ufficiale}
    \label{risk:android-dev} 
\end{risk}

\subsection{Rischi specifici}

\begin{risk}{Inadeguatezza della libreria \emph{PCL} e relativa documentazione}
	\riskprob{Bassa}
	\riskimpact{Basso}
    \riskdescription{La libreria \emph{PCL} espone molti algoritmi general-purpose per la manipolazione di Point Cloud, che non necessariamente si adattano alle necessità specifiche del progetto in questione}
    \risksolution{La libreria espone anche molte funzionalità basilari che è possibile utilizzare e combinare per ovviare alle necessità che algoritmi specifici non sempre soddisfano}
    \label{risk:PCL-issues} 
\end{risk}

\begin{risk}{Impredicibilità dei Point Cloud}
	\riskprob{Alta}
	\riskimpact{Medio}
    \riskdescription{La qualità di un oggetto ricostruito come Point Cloud è impredicibile, a causa dei fenomeni di \emph{drifting}, \emph{ghosting} ed \glspl{artefatto} di cui soffre la ricostruzione. Inoltre la tipologia di oggetti scansionati è altamente variabile}
    \risksolution{Il processo di filtraggio dev'essere attentamente impostato in modo da trattare al meglio tali imperfezioni e tipologie di oggetti. Bisogna inevitabilmente prendere in considerazione che non sempre il filtraggio può ottenere i risultati voluti, e richiedere quindi una nuova ricostruzione all'utente}
    \label{risk:pc-quality} 
\end{risk}

\begin{risk}{Difficoltà nella comunicazione client-server}
	\riskprob{Media}
	\riskimpact{Basso}
    \riskdescription{L'applicazione verrà usata in futuro sul campo da addetti aziendali, che potrebbero non disporre di una connessione stabile con il \emph{server}}
    \risksolution{Bisogna prevedere e controllare una possibile assenza di connessione e regolarsi di conseguenza, fornendo ad esempio all'utente la possibilità di salvare il proprio lavoro e ritentare la comunicazione col \emph{server} in un secondo momento. Dato lo stadio prototipale del progetto e l'utilizzo indoor dell'applicativo, il rischio è altamente improbabile}
    \label{risk:connection-refused} 
\end{risk}

\begin{risk}{Difficoltà nel calcolo del volume}
	\riskprob{Alta}
	\riskimpact{Media}
    \riskdescription{Il calcolo del volume dipende fortemente dal risultato dell'elaborazione attraverso le funzionalità della \emph{PCL}, in particolare dall'assenza di punti estranei all'oggetto prima di effettuarne il meshing, al quale segue il calcolo.}
    \risksolution{Lo studente deve eseguire una \emph{suite} di test adeguati per valutare la bontà del calcolo del volume e la percentuale d'errore ammissibile, e capire le cause che portano il calcolo a divergere dal volume reale}
    \label{risk:} 
\end{risk}