% !TEX encoding = UTF-8
% !TEX TS-program = pdflatex
% !TEX root = ../tesi.tex
% !TEX spellcheck = it-IT

%**************************************************************
\chapter{Studio di fattibilità ed Analisi dei rischi}
\label{cap:descrizione-stage}
%**************************************************************

\intro{Il progetto si è subito presentato come sperimentale, impegnativo e facilmente soggetto a fallimento. Per questo si è reso necessario studiarne attentamente la fattibilità e valutarne i rischi}\\

%**************************************************************
\section{Introduzione al progetto}

Data la natura innovativa del progetto è stato necessario produrre diversi prototipi ed
effettuare l’Analisi dei Rischi e lo Studio di Fattibilità in diverse fasi.
Seguendo quest'approccio è stato possibile valutare le potenzialità e i limiti del \emph{device Tango} e delle API annesse, e considerare l'applicabilità degli algoritmi general purpose della libreria PCL al caso particolare del progetto in questione.

%**************************************************************
\section{Studio di fattibilità}
Prima di iniziare il progetto è stato effettuato un accurato studio di fattibilità basato sullo studio della libreria PCL e della vasta quantità di esempi d'utilizzo della stessa reperibili online. Si sono cercate poi soluzioni per il calcolo del volume di una mesh.\\
Per il lato client si sono esplorate soluzioni per la visualizzazione di mesh salvate in formato OBJ e per la comunicazione client-server.

\subsection{Funzionalità per il trattamento di Point Cloud}
\'E parso subito evidente che l'applicativo necessita di un formato standard, trasferibile via Web, per il trattamento di Point Cloud, e delle funzionalità di I/O associate. A tal riguardo è risultata particolarmente utile la documentazione della PCL:
\begin{itemize}
\item\textbf{The PCD (Point Cloud Data) file format\footcite{http://pointclouds.org/documentation/tutorials/pcd_file_format.php/pcd-file-format}}:
un formato facilmente comprensibile e portabile per definire nuvole di punti
\item\textbf{Reading Point Cloud data from PCD files\footcite{http://pointclouds.org/documentation/tutorials/reading_pcd.php/reading-pcd}}:
tutorial che espone le funzionalità di lettura e caricamento di un file PCD
\item\textbf{Writing Point Cloud data to PCD files\footcite{http://pointclouds.org/documentation/tutorials/writing_pcd.php/writing-pcd}}:
tutorial che espone le funzionalità di salvataggio di un file PCD
\end{itemize}

\subsection{Funzionalità per il filtraggio di Point Cloud}
Sono disponibili sul sito di PCL svariati esempi open-source che utilizzano le potenzialità della libreria per filtrare un Point Cloud, ad esempio:
\begin{itemize}
\item\textbf{Filtering a PointCloud using a PassThrough filter\footcite{http://pointclouds.org/documentation/tutorials/passthrough.php/passthrough}}:
tutorial per la creazione di un filtro capace di eliminare i punti che non rientrano in un certo range definito dall'utente.
\item\textbf{Downsampling a PointCloud using a VoxelGrid filter\footcite{http://pointclouds.org/documentation/tutorials/voxel_grid.php/voxelgrid}}:
tutorial per la creazione di un filtro capace di effettuare il downsampling di un PointCLoud
\item\textbf{Removing outliers using a StatisticalOutlierRemoval filter\footcite{http://pointclouds.org/documentation/tutorials/statistical_outlier.php/statistical-outlier-removal}}:
tutorial per la creazione di un filtro capace di eliminare i punti isolati, quindi non interessanti, di un Point Cloud
\item\textbf{Plane model segmentation\footcite{http://pointclouds.org/documentation/tutorials/planar_segmentation.php/planar-segmentation}}:
tutorial per la creazione di un filtro capace di eliminare le componenti planari di un Point Cloud.
\item\textbf{Euclidean Cluster Extraction\footcite{http://pointclouds.org/documentation/tutorials/cluster_extraction.php/cluster-extraction}}:
tutorial per la creazione di un filtro capace di isolare ed estrarre gruppi consistenti e distinguibili di punti da un Point Cloud.
\end{itemize}

\subsection{Funzionalità per il meshing}
Sono disponibili sul sito di PCL alcuni esempi open-source che utilizzano le potenzialità della libreria per generare una mesh tridimensionale a partire da un Point Cloud, ad esempio:
\begin{itemize}
\item \textbf{Fast triangulation of unordered point clouds\footcite{http://pointclouds.org/documentation/tutorials/greedy_projection.php/greedy-triangulation}}:
tutorial per la creazione di una mesh triangolare a partire da una nuvola di punti
\item \textbf{Fitting trimmed B-splines to unordered point clouds\footcite{http://pointclouds.org/documentation/tutorials/bspline_fitting.php/bspline-fitting}}:
tutorial per la creazione di una mesh regolare adattata iterativamente alla nuvola di punti
\end{itemize}

\newpage
\subsection{Metodologie per il calcolo del volume}
Durante la ricerca di un affidabile metodo matematico per il calcolo del volume di una mesh è risultato che il più adatto, validato ormai da anni di utilizzo, è il calcolo del volume con segno di una mesh triangolare come descritto ad esempio nel paper "EFFICIENT FEATURE EXTRACTION FOR 2D/3D OBJECTS IN MESH REPRESENTATION \footcite{http://research.microsoft.com/en-us/um/people/chazhang/publications/icip01_ChaZhang.pdf}" da Cha Zhang e Tsuhan Chen.

\subsection{Visualizzazione di file in formato OBJ}
Per visualizzare le mesh prodotte dal server su tablet è stato necessario ricercare esempi di applicativi che permettessero di trattare oggetti tridimensionali in formato OBJ. A tal riguardo si sono studiate due applicazioni:
\begin{itemize}
\item\textbf{3D viewer\footcite{https://play.google.com/store/apps/details?id=com.pcvirt.Viewer3D&hl=it}}: 
applicazione android per la visualizzazione di file OBJ e 3DS
\item\textbf{3D Model Viewer Open Source\footcite{https://play.google.com/store/apps/details?id=org.andresoviedo.dddmodel&hl=it}}:
applicazione open-source android per la visualizzazione di file OBJ realizzata da Andres Oviedo, il cui codice è liberamente reperibile dalla repository\footcite{https://github.com/andresoviedo/android-3D-model-viewer} GitHub
\end{itemize}

\subsection{Comunicazione client-server}
Per implementare la comunicazione client-server la scelta più logica è stata di seguire esplorare la documentazione ufficiale Android, che mette a disposizione il package "\textbf{Apache HTTP\footcite{https://developer.android.com/reference/org/apache/http/package-summary.html}}" che risponde alle necessità richieste.

\subsection{Conclusioni}
Alla luce della disponibilità e qualità degli esempi sopracitati il progetto è apparso fattibile e quindi è stato possibile dare il via al ciclo di vita del progetto software.


\newpage
%**************************************************************
\section{Analisi preventiva dei rischi}

Durante la fase di analisi iniziale sono stati individuati alcuni possibili rischi a cui si potrà andare incontro.
Si è quindi proceduto a elaborare delle possibili soluzioni per far fronte a tali rischi.\\

\subsection{Rischi generali}

\begin{risk}{Limiti fisici del tablet}
	\riskprob{Alta}
	\riskimpact{Medio}
    \riskdescription{Il dispositivo è dotato di sensori IR che sfruttano la riflessione della luce per determinare la distanza dei punti che è in grado di individuare, e di fotocamere RGB. Superfici lucide/riflettenti o tendenti al nero possono compromettere gravemente la qualità della misurazione, così come gli ambienti con illuminazione scarsa o troppo intensa}
    \risksolution{ Attento studio delle potenzialità dell'hardware del tablet\footcite{https://developers.google.com/tango/hardware/tablet} e del loro utilizzo\footcite{https://developers.google.com/tango/overview/depth-perception} nelle Tango API}
    \label{risk:device-limits} 
\end{risk}

\begin{risk}{Incomprensibilità dello stato del progetto}
	\riskprob{Media}
	\riskimpact{Medio}
    \riskdescription{Venendo inserito in un progetto già avviato, lo studente necessita di comprendere a fondo lo stato dell'arte dell'applicativo}
    \risksolution{Lo studente deve, consultando la documentazione creata e collaborando strettamente con gli altri elementi del team e con il tutor aziendale, assicurasi di comprendere quanto è stato fatto, come è stato fatto e le necessità future del progetto.}
    \label{risk:project-status} 
\end{risk}

\begin{risk}{Scarsa competenza nello sviluppo Android}
	\riskprob{Alta}
	\riskimpact{Basso}
    \riskdescription{Lo studente ha scarse conoscenze dello sviluppo Android, ma conosce il linguaggio Java}
    \risksolution{Lo studente deve effettuare un adeguato periodo di studio ed assumere familiarità con la documentazione ufficiale}
    \label{risk:android-dev} 
\end{risk}

\subsection{Rischi specifici}

\begin{risk}{Inadeguatezza della libreria PCL e relativa documentazione}
	\riskprob{Bassa}
	\riskimpact{Basso}
    \riskdescription{La libreria PCL espone molti algoritmi general-purpose per la manipolazione di Point Cloud, che non necessariamente si adattano alle necessità specifiche del progetto in questione}
    \risksolution{La libreria espone anche molte funzionalità basilari che è possibile utilizzare e combinare per ovviare alle necessità che algoritmi specifici non sempre soddisfano}
    \label{risk:PCL-issues} 
\end{risk}

\begin{risk}{Impredicibilità dei Point Cloud}
	\riskprob{Alta}
	\riskimpact{Medio}
    \riskdescription{La qualità di un oggetto ricostruito come Point Cloud è impredicibile, a causa dei fenomeni di \emph{drifting}, \emph{ghosting} ed artefatti di cui soffre la ricostruzione. Inoltre la tipologia di oggetti scansionati è altamente variabile}
    \risksolution{Il processo di filtraggio dev'essere attentamente impostato in modo da trattare al meglio tali imperfezioni e tipologie di oggetti. Bisogna inevitabilmente prendere in considerazione che non sempre il filtraggio può ottenere i risultati voluti, e richiedere quindi una nuova ricostruzione all'utente}
    \label{risk:pc-quality} 
\end{risk}

\begin{risk}{Difficoltà nella comunicazione client-server}
	\riskprob{Media}
	\riskimpact{Basso}
    \riskdescription{L'applicazione verrà usata in futuro sul campo da addetti aziendali, che potrebbero non disporre di una connessione stabile con il server}
    \risksolution{Bisogna prevedere e controllare una possibile assenza di connessione e regolarsi di conseguenza, fornendo ad esempio all'utente la possibilità di salvare il proprio lavoro e ritentare la comunicazione col server in un secondo momento. Dato lo stadio prototipale del progetto e l'utilizzo indoor dell'applicativo, il rischio è altamente improbabile}
    \label{risk:connection-refused} 
\end{risk}

\begin{risk}{Difficoltà nel calcolo del volume}
	\riskprob{Alta}
	\riskimpact{Media}
    \riskdescription{Il calcolo del volume dipende fortemente dal risultato dell'elaborazione attraverso le funzionalità della PCL, in particolare dall'assenza di punti estranei all'oggetto prima di effettuarne il meshing, al quale segue il calcolo.}
    \risksolution{Lo studente deve eseguire una \emph{suite} di test adeguati per valutare la bontà del calcolo del volume e la percentuale d'errore ammissibile, e capire le cause che portano il calcolo a divergere dal volume reale}
    \label{risk:} 
\end{risk}