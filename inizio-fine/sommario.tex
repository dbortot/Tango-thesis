% !TEX encoding = UTF-8
% !TEX TS-program = pdflatex
% !TEX root = ../tesi.tex
% !TEX spellcheck = it-IT

%**************************************************************
% Sommario
%**************************************************************
\cleardoublepage
\phantomsection
\pdfbookmark{Sommario}{Sommario}
\begingroup
\let\clearpage\relax
\let\cleardoublepage\relax
\let\cleardoublepage\relax

\chapter*{Sommario}

Il presente documento descrive il lavoro svolto durante il periodo di stage, della durata di circa trecentoventi ore, dal laureando Davide Bortot presso l'azienda VIC S.r.l.
Obbiettivo del tirocinio era lo sviluppo di un'applicazione prototipale per il \emph{tablet Project Tango} per l'aquisizione di scansioni di oggetti reali sotto forma di 'Nuvola di Punti'. \\
In primo luogo era richiesto il filtraggio e l'elaborazione dei dati acquisiti in modo da isolare l'oggetto scansionato, e successivamente di creare una \emph{mesh} tridimensionale a partire dai punti rimanenti. Scopo finale era calcolare il volume dell'oggetto tridimensionale così ottenuto.

%\vfill
%
%\selectlanguage{english}
%\pdfbookmark{Abstract}{Abstract}
%\chapter*{Abstract}
%
%\selectlanguage{italian}

\endgroup			

\vfill

