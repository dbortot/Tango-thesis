
%**************************************************************
% Acronimi
%**************************************************************
\renewcommand{\acronymname}{Acronimi e abbreviazioni}

\newacronym[description={\glslink{apig}{Application Program Interface}}]
    {api}{API}{Application Program Interface}

\newacronym[description={\glslink{umlg}{Unified Modeling Language}}]
    {uml}{UML}{Unified Modeling Language}
       
\newacronym[description={\glslink{PCL}{Point Cloud Library}}]
    {pcl}{PCL}{Point Cloud Library}     

%**************************************************************
% Glossario
%**************************************************************
%\renewcommand{\glossaryname}{Glossario}

\newglossaryentry{PCL}
{
    name=\glslink{PCL}{PCL},
    text=PCL,
    sort=PCL,
    description={Una  delle librerie di maggior rilievo nel campo della \emph{Computer Vision}, \emph{Open Source}, scritta in C++, di algoritmi per l'elaborazione di \emph{Point Cloud} tridimensionali. Contiene algoritmi per il filtraggio, la segmentazione, la \emph{registration} e il \emph{meshing} di Point Cloud, per citarne alcuni. \\
La libreria è ampiamente utilizzata da qualsiasi applicativo debba trattare nuvole di punti, ed oltre ad essere utile ed efficiente è anche ben documentata, con molti esempi d'utilizzo reperibili online.\\
Le notevoli potenzialità della libreria sono state utilizzate intensivamente nell'applicativo con lo scopo di isolare i punti appartenenti all'oggetto scansionato dal resto del Point Cloud, ed effettuarne quindi il meshing.}
}

\newglossaryentry{umlg}
{
    name=\glslink{uml}{UML},
    text=UML,
    sort=uml,
    description={in ingegneria del software \emph{UML, Unified Modeling Language} (ing. linguaggio di modellazione unificato) è un linguaggio di modellazione e specifica basato sul paradigma object-oriented. L'\emph{UML} svolge un'importantissima funzione di ``lingua franca'' nella comunità della progettazione e programmazione a oggetti. Gran parte della letteratura di settore usa tale linguaggio per descrivere soluzioni analitiche e progettuali in modo sintetico e comprensibile a un vasto pubblico}
}

\newglossaryentry{Point Cloud} {
	name = \glslink{Point Cloud}{Point Cloud},
	text = Point Cloud,
	sort = Point Cloud,
	description = {Un \emph{Point Cloud} è modello matematico che descrive un oggetto tridimensionale semplicemente come insieme del maggior numero possibile di punti dell'oggetto stesso.\\
È molto usato in ambito di \emph{Computer vision} e realtà aumentata}
}


\newglossaryentry{API} {
	name = \glslink{API}{API},
	text = API,
	sort = API,
	description = {Con \emph{API} o \emph{Application Programming Interface} si intende ogni insieme di metodi e procedure resi disponibili al programmatore. Di solito raggruppati a formare un set di strumenti specifici per l'esecuzione di un determinato compito}
}

\newglossaryentry{artefatto} {
	name = \glslink{artefatto}{Artefatto},
	text = artefatto,
	plural = artefatti,
	sort = artefatto,
	description = {Gli \emph{artefatti} sono degli elementi presenti all'interno di una ricostruzione \emph{Point Cloud} ma non nella realtà, dovuti errori nella rilevazione. Sono generalmente di forma planare e sospesi qualche centimetro al di sopra del pavimento}
}

\newglossaryentry{Design Pattern} {
	name = \glslink{Design Pattern}{Design Pattern},
	text = Design Pattern,
	sort = design pattern,
	description = {Un \emph{Design Pattern} è una soluzione progettuale di comprovata bontà ad un problema ricorrente in un certo contesto}
}

\newglossaryentry{ghosting} {
	name = \glslink{ghosting}{Ghosting},
	text = ghosting,
	sort = ghosting,
	description = {Il problema del \emph{Ghosting} affligge alcune registrazioni effettuate con il prodotto: è dovuto ad una errata stima della posizione del dispositivo e produce ricostruzioni tridimensionali affette da errori. Se visualizzate graficamente il tali ricostruzioni presentano l'oggetto come se fosse sdoppiato, ad esempio come in figura \ref{fig:no_drift_correction}}
}

\newglossaryentry{feature} {
	name = \glslink{feature}{Feature},
	text = feature,
	sort = feature,
	description = {Una \emph{feature} è un punto particolare nello spazio che il processo di \emph{Area Learning} ritiene facilmente riconoscibile, e che quindi è salvato ed usato come punto di riferimento in un determinato ambiente. Un'area ha generalmente qualche centinaio di \emph{feature}}
}

\newglossaryentry{Fisheye} {
	name = \glslink{Fisheye}{Fotocamera Fisheye},
	text = fotocamera Fisheye,
	sort = Fotocamera Fisheye,
	description = {Una Fotocamera \emph{Fish-eye} dispone di un obiettivo grandangolare che abbraccia un angolo di campo di circa 180 gradi, in particolare quella in dotazione nel dispositivo usato è in bianco e nero}
}	

\newglossaryentry{mesh} {
	name = \glslink{mesh}{Mesh},
	text = mesh,
	plural = meshes,
	sort = Mesh,
	description = {Una \emph{mesh} o \emph{mesh} poligonale è una collezione di vertici, spigoli e facce che definiscono la forma di un oggetto tridimensionale}
}	

\newglossaryentry{Motion Tracking} {
	name = \glslink{Motion Tracking}{Motion Tracking},
	text = Motion Tracking,
	sort = Motion Tracking,
	description = {Il \emph{Motion Tracking} è una tecnica che permette al dispositivo di tracciare i suoi movimenti all'interno di un area}
}	

\newglossaryentry{drifting} {
	name = \glslink{drifting}{Drifting},
	text = drifting,
	sort = Drifting,
	description = {Il \emph{drifting} è l'errore di calcolo nella posizione di un dispositivo che utilizza il \emph{Motion Tracking}, che ne causa una scorretta stima della posizione.}
}	

\newglossaryentry{Project Tango} {
	name = \glslink{Project Tango}{Project Tango},
	text = Project Tango,
	sort = Project Tango,
	description = {Il progetto \emph{Tango} è un progetto promosso e portato avanti da \emph{Google}. Si propone di promuovere nuovi tipi di dispositivi dotati di \emph{Hardware} innovativo, come fotocamera \emph{Fish-Eye}, sensore di profondità etc. Grazie ai loro sensori i dispositivi \emph{Tango} sono in grado di avere una certa conoscenza dell'ambiente che li circonda: sono in grado di tracciare il proprio movimento ed orientazione, di ricordare gli ambienti dove sono già stati, di avere una visione tridimensionale degli oggetti}
}	
%
\newglossaryentry{voxel} {
	name = \glslink{voxel}{Voxel},
	text = voxel,
	sort = Voxel,
	description = {Un \emph{voxel}, detto anche \emph{pixel} volumetrico, è un elemento di volume che rappresenta un valore di intensità di segnale o di colore in uno spazio tridimensionale}
}	
%




