
%**************************************************************
% Acronimi
%**************************************************************
\renewcommand{\acronymname}{Acronimi e abbreviazioni}

\newacronym[description={\glslink{apig}{Application Program Interface}}]
    {api}{API}{Application Program Interface}

\newacronym[description={\glslink{umlg}{Unified Modeling Language}}]
    {uml}{UML}{Unified Modeling Language}

%**************************************************************
% Glossario
%**************************************************************
%\renewcommand{\glossaryname}{Glossario}

\newglossaryentry{apig}
{
    name=\glslink{api}{API},
    text=Application Program Interface,
    sort=api,
    description={in informatica con il termine \emph{Application Programming Interface API} (ing. interfaccia di programmazione di un'applicazione) si indica ogni insieme di procedure disponibili al programmatore, di solito raggruppate a formare un set di strumenti specifici per l'espletamento di un determinato compito all'interno di un certo programma. La finalità è ottenere un'astrazione, di solito tra l'hardware e il programmatore o tra software a basso e quello ad alto livello semplificando così il lavoro di programmazione}
}

\newglossaryentry{umlg}
{
    name=\glslink{uml}{UML},
    text=UML,
    sort=uml,
    description={in ingegneria del software \emph{UML, Unified Modeling Language} (ing. linguaggio di modellazione unificato) è un linguaggio di modellazione e specifica basato sul paradigma object-oriented. L'\emph{UML} svolge un'importantissima funzione di ``lingua franca'' nella comunità della progettazione e programmazione a oggetti. Gran parte della letteratura di settore usa tale linguaggio per descrivere soluzioni analitiche e progettuali in modo sintetico e comprensibile a un vasto pubblico}
}

\newglossaryentry{pc} {
	name = \glslink{point cloud}{Point Cloud},
	text = Point Cloud,
	sort = point cloud,
	description = {Un \emph{Point Cloud} è modello matematico che descrive un oggetto tridimensionale semplicemente come insieme del maggior numero possibile di punti dell'oggetto stesso.\\
È molto usato in ambito di \emph{Computer vision} e realtà aumentata}
}


\newglossaryentry{API} {
	name = \glslink{API}{api},
	text = API,
	sort = api,
	description = {Con \emph{API} o \emph{Application Programming Interface} si intende ogni insieme di metodi e procedure resi disponibili al programmatore. Di solito raggruppati in a formare un set di strumenti specifici per l'esecuzione di un determinato compito.}
}

\newglossaryentry{Artefatto} {
	name = \glslink{Artefatto}{artefatto},
	text = Artefatto,
	sort = artefatto,
	description = {Gli \emph{artefatti} sono degli elementi presenti all'interno di una ricostruzione \emph{Point Cloud} ma non nella realtà, dovuti errori nella rilevazione. Sono generalmente di forma planare e sospesi qualche centimetro al di sopra del pavimento.}
}

\newglossaryentry{Design Pattern} {
	name = \glslink{Design Pattern}{design pattern},
	text = Design Pattern,
	sort = design pattern,
	description = {Un \emph{Design Pattern} è una soluzione progettuale di comprovata bontà ad un problema ricorrente in un certo contesto.}
}

%\subsubsection{Ghosting}
%Il problema del \emph{Ghosting} affligge alcune registrazioni effettuate con il prodotto: è dovuto ad una errata stima della posizione del dispositivo e produce ricostruzioni tridimensionali affette da errori. Se visualizzate graficamente il tali ricostruzioni presentano l'oggetto come se fosse sdoppiato, ad esempio come in figura \ref{figure:pcloud_ghosting}.
%
%\subsubsection{Feature}
%Una \emph{feature} è un punto particolare nello spazio che il processo di \emph{Area Learning} ritiene facilmente riconoscibile, e che quindi è salvato ed usato come punto di riferimento in un determinato ambiente. Un'area ha generalmente qualche centinaio di \emph{feature}.
%
%\subsubsection{Fotocamera Fish-eye}
%Una Fotocamera \emph{Fish-eye} è un obbiettivo grandangolare che abbraccia un angolo di campo di circa 180 gradi, in particolare quello in dotazione nel dispositivo usato è in bianco e nero.
%
%\subsubsection{Mesh}
%Una \emph{mesh} o \emph{mesh} poligonale è una collezione di vertici, spigoli e facce che definiscono la forma di un oggetto tridimensionale.
%
%\subsubsection{Milestone}
%Letteralmente pietre miliari, vengono tipicamente utilizzate nella pianificazione e gestione
%di progetti complessi al fine di indicare importanti traguardi intermedi nello svolgimento del progetto
%stesso. Molto spesso sono rappresentate da eventi, cioè da attività con durata zero o di un giorno,
%e vengono evidenziate in maniera diversa dalle altre attività nell'ambito dei documenti di progetto.
%
%\subsubsection{Motion Tracking}
%Il \emph{Motion Tracking} è una tecnica che permette al dispositivo di tracciare i suoi movimenti all'interno di un area.
%
%\subsubsection{Quaternione}
%Un quaternione è un oggetto matematico del tipo
%$$a+bi+cj+dk$$
%dove $a,b,c,d\in \mathbb{R}$ e $i$, $j$ e $k$ sono simboli che si comportano come l'unità immaginaria dell'insieme dei numeri complessi $\mathbb{C}$.\\
%Una loro possibile applicazione è la modellazione di rotazioni nello spazio, per questo sono molto usati in fisica teorica, ma anche in \emph{Computer} grafica e in robotica.
%
%\subsubsection{Point Cloud}
%Un \emph{Point Cloud} è modello matematico che descrive un oggetto tridimensionale semplicemente come insieme del maggior numero possibile di punti dell'oggetto stesso.\\
%È molto usato in ambito di \emph{Computer vision} e realtà aumentata.
%
%\subsubsection{Race Condition}
%Una \emph{Race Condition} è un fenomeno proprio dei sistemi concorrenti ed avviene quando il risultato finale dell'elaborazione dipende dalla temporizzazione o dalla sequenza con cui vengono eseguiti i processi. 
%
%\subsubsection{Stakeholder}
%Gli \emph{Stakeholder}, o portatori di interesse, sono l'insieme delle persone che a vario titolo sono coinvolte nel ciclo di vita del \emph{Software} con influenza sul prodotto.
%
%\subsubsection{Tango Project}
%Il progetto \emph{Tango} è un progetto promosso e portato avanti da \emph{Google}. Si propone di promuovere nuovi tipi di dispositivi dotati di \emph{Hardware} innovativo, come fotocamera \emph{Fish-Eye}, sensore di profondità etc. Grazie ai loro sensori i dispositivi \emph{Tango} sono in grado di avere una certa conoscenza dell'ambiente che li circonda: sono in grado di tracciare il proprio movimento ed orientazione, di ricordare gli ambienti dove sono già stati, di avere una visione tridimensionale degli oggetti.
%
%\subsubsection{UML - Unified Modelling Language}
%Famiglia di notazioni grafiche che si basano su un singolo meta-modello e servono a
%supportare la descrizione e il progetto dei sistemi software.
%
%\subsubsection{Voxel}
%Un \emph{voxel}, detto anche \emph{pixel} volumetrico, è un elemento di volume che rappresenta un valore di intensità di segnale o di colore in uno spazio tridimensionale.
%




