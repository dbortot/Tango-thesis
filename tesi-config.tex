%**************************************************************
% file contenente le impostazioni della tesi
%**************************************************************

%**************************************************************
% Frontespizio
%**************************************************************
\newcommand{\myName}{Davide Bortot}                             % autore
\newcommand{\myTitle}{Project Tango e PCL: un'applicazione pratica alle 'Nuvole di Punti'}                    	% titolo tesi
\newcommand{\myDegree}{Tesi di laurea}                % tipo di tesi
\newcommand{\myUni}{Università degli Studi di Padova}           % università
\newcommand{\myFaculty}{Corso di Laurea in Informatica}         % facoltà
\newcommand{\myDepartment}{Dipartimento di Matematica}          % dipartimento
\newcommand{\myProf}{Ombretta Gaggi}                            % relatore
\newcommand{\myLocation}{Padova}                                % dove
\newcommand{\myAA}{2015-2016}                                   % anno accademico
\newcommand{\myTime}{Oct 2016}                                  % quando


%**************************************************************
% Impostazioni di impaginazione
% see: http://wwwcdf.pd.infn.it/AppuntiLinux/a2547.htm
%**************************************************************

\setlength{\parindent}{14pt}   % larghezza rientro della prima riga
\setlength{\parskip}{0pt}   % distanza tra i paragrafi


%**************************************************************
% Impostazioni di biblatex
%**************************************************************
\bibliography{bibliografia} % database di biblatex 

\defbibheading{bibliography}
{
    \cleardoublepage
    \phantomsection 
    \addcontentsline{toc}{chapter}{\bibname}
    \chapter*{\bibname\markboth{\bibname}{\bibname}}
}

\setlength\bibitemsep{1.5\itemsep} % spazio tra entry

\DeclareBibliographyCategory{opere}
\DeclareBibliographyCategory{web}

\addtocategory{opere}{womak:lean-thinking}
\addtocategory{web}{site:agile-manifesto}

\defbibheading{opere}{\section*{Riferimenti bibliografici}}
\defbibheading{web}{\section*{Siti Web consultati}}


%**************************************************************
% Impostazioni di caption
%**************************************************************
\captionsetup{
    tableposition=top,
    figureposition=bottom,
    font=small,
    format=hang,
    labelfont=bf
}

%**************************************************************
% Impostazioni di glossaries
%**************************************************************

%**************************************************************
% Acronimi
%**************************************************************
\renewcommand{\acronymname}{Acronimi e abbreviazioni}

\newacronym[description={\glslink{apig}{Application Program Interface}}]
    {api}{API}{Application Program Interface}

\newacronym[description={\glslink{umlg}{Unified Modeling Language}}]
    {uml}{UML}{Unified Modeling Language}
       
\newacronym[description={\glslink{PCL}{Point Cloud Library}}]
    {pcl}{PCL}{Point Cloud Library}     

%**************************************************************
% Glossario
%**************************************************************
%\renewcommand{\glossaryname}{Glossario}

\newglossaryentry{PCL}
{
    name=\glslink{PCL}{PCL},
    text=PCL,
    sort=PCL,
    description={Una  delle librerie di maggior rilievo nel campo della \emph{Computer Vision}, \emph{Open Source}, scritta in C++, di algoritmi per l'elaborazione di \emph{Point Cloud} tridimensionali. Contiene algoritmi per il filtraggio, la segmentazione, la \emph{registration} e il \emph{meshing} di Point Cloud, per citarne alcuni. \\
La libreria è ampiamente utilizzata da qualsiasi applicativo debba trattare nuvole di punti, ed oltre ad essere utile ed efficiente è anche ben documentata, con molti esempi d'utilizzo reperibili online.\\
Le notevoli potenzialità della libreria sono state utilizzate intensivamente nell'applicativo con lo scopo di isolare i punti appartenenti all'oggetto scansionato dal resto del Point Cloud, ed effettuarne quindi il meshing.}
}

\newglossaryentry{umlg}
{
    name=\glslink{uml}{UML},
    text=UML,
    sort=uml,
    description={in ingegneria del software \emph{UML, Unified Modeling Language} (ing. linguaggio di modellazione unificato) è un linguaggio di modellazione e specifica basato sul paradigma object-oriented. L'\emph{UML} svolge un'importantissima funzione di ``lingua franca'' nella comunità della progettazione e programmazione a oggetti. Gran parte della letteratura di settore usa tale linguaggio per descrivere soluzioni analitiche e progettuali in modo sintetico e comprensibile a un vasto pubblico}
}

\newglossaryentry{Point Cloud} {
	name = \glslink{Point Cloud}{Point Cloud},
	text = Point Cloud,
	sort = Point Cloud,
	description = {Un \emph{Point Cloud} è modello matematico che descrive un oggetto tridimensionale semplicemente come insieme del maggior numero possibile di punti dell'oggetto stesso.\\
È molto usato in ambito di \emph{Computer vision} e realtà aumentata}
}


\newglossaryentry{API} {
	name = \glslink{API}{API},
	text = API,
	sort = API,
	description = {Con \emph{API} o \emph{Application Programming Interface} si intende ogni insieme di metodi e procedure resi disponibili al programmatore. Di solito raggruppati a formare un set di strumenti specifici per l'esecuzione di un determinato compito}
}

\newglossaryentry{artefatto} {
	name = \glslink{artefatto}{Artefatto},
	text = artefatto,
	plural = artefatti,
	sort = artefatto,
	description = {Gli \emph{artefatti} sono degli elementi presenti all'interno di una ricostruzione \emph{Point Cloud} ma non nella realtà, dovuti errori nella rilevazione. Sono generalmente di forma planare e sospesi qualche centimetro al di sopra del pavimento}
}

\newglossaryentry{Design Pattern} {
	name = \glslink{Design Pattern}{Design Pattern},
	text = Design Pattern,
	sort = design pattern,
	description = {Un \emph{Design Pattern} è una soluzione progettuale di comprovata bontà ad un problema ricorrente in un certo contesto}
}

\newglossaryentry{ghosting} {
	name = \glslink{ghosting}{Ghosting},
	text = ghosting,
	sort = ghosting,
	description = {Il problema del \emph{Ghosting} affligge alcune registrazioni effettuate con il prodotto: è dovuto ad una errata stima della posizione del dispositivo e produce ricostruzioni tridimensionali affette da errori. Se visualizzate graficamente il tali ricostruzioni presentano l'oggetto come se fosse sdoppiato, ad esempio come in figura \ref{fig:no_drift_correction}}
}

\newglossaryentry{feature} {
	name = \glslink{feature}{Feature},
	text = feature,
	sort = feature,
	description = {Una \emph{feature} è un punto particolare nello spazio che il processo di \emph{Area Learning} ritiene facilmente riconoscibile, e che quindi è salvato ed usato come punto di riferimento in un determinato ambiente. Un'area ha generalmente qualche centinaio di \emph{feature}}
}

\newglossaryentry{Fisheye} {
	name = \glslink{Fisheye}{Fotocamera Fisheye},
	text = fotocamera Fisheye,
	sort = Fotocamera Fisheye,
	description = {Una Fotocamera \emph{Fish-eye} dispone di un obiettivo grandangolare che abbraccia un angolo di campo di circa 180 gradi, in particolare quella in dotazione nel dispositivo usato è in bianco e nero}
}	

\newglossaryentry{mesh} {
	name = \glslink{mesh}{Mesh},
	text = mesh,
	plural = meshes,
	sort = Mesh,
	description = {Una \emph{mesh} o \emph{mesh} poligonale è una collezione di vertici, spigoli e facce che definiscono la forma di un oggetto tridimensionale}
}	

\newglossaryentry{Motion Tracking} {
	name = \glslink{Motion Tracking}{Motion Tracking},
	text = Motion Tracking,
	sort = Motion Tracking,
	description = {Il \emph{Motion Tracking} è una tecnica che permette al dispositivo di tracciare i suoi movimenti all'interno di un area}
}	

\newglossaryentry{drifting} {
	name = \glslink{drifting}{Drifting},
	text = drifting,
	sort = Drifting,
	description = {Il \emph{drifting} è l'errore di calcolo nella posizione di un dispositivo che utilizza il \emph{Motion Tracking}, che ne causa una scorretta stima della posizione.}
}	

\newglossaryentry{Project Tango} {
	name = \glslink{Project Tango}{Project Tango},
	text = Project Tango,
	sort = Project Tango,
	description = {Il progetto \emph{Tango} è un progetto promosso e portato avanti da \emph{Google}. Si propone di promuovere nuovi tipi di dispositivi dotati di \emph{Hardware} innovativo, come fotocamera \emph{Fish-Eye}, sensore di profondità etc. Grazie ai loro sensori i dispositivi \emph{Tango} sono in grado di avere una certa conoscenza dell'ambiente che li circonda: sono in grado di tracciare il proprio movimento ed orientazione, di ricordare gli ambienti dove sono già stati, di avere una visione tridimensionale degli oggetti}
}	
%
\newglossaryentry{voxel} {
	name = \glslink{voxel}{Voxel},
	text = voxel,
	sort = Voxel,
	description = {Un \emph{voxel}, detto anche \emph{pixel} volumetrico, è un elemento di volume che rappresenta un valore di intensità di segnale o di colore in uno spazio tridimensionale}
}	
%




 % database di termini
\makeglossaries


%**************************************************************
% Impostazioni di graphicx
%**************************************************************
\graphicspath{{immagini/}} % cartella dove sono riposte le immagini


%**************************************************************
% Impostazioni di hyperref
%**************************************************************
\hypersetup{
    %hyperfootnotes=false,
    %pdfpagelabels,
    %draft,	% = elimina tutti i link (utile per stampe in bianco e nero)
    colorlinks=true,
    linktocpage=true,
    pdfstartpage=1,
    pdfstartview=FitV,
    % decommenta la riga seguente per avere link in nero (per esempio per la stampa in bianco e nero)
    %colorlinks=false, linktocpage=false, pdfborder={0 0 0}, pdfstartpage=1, pdfstartview=FitV,
    breaklinks=true,
    pdfpagemode=UseNone,
    pageanchor=true,
    pdfpagemode=UseOutlines,
    plainpages=false,
    bookmarksnumbered,
    bookmarksopen=true,
    bookmarksopenlevel=1,
    hypertexnames=true,
    pdfhighlight=/O,
    %nesting=true,
    %frenchlinks,
    urlcolor=webbrown,
    linkcolor=RoyalBlue,
    citecolor=webgreen,
    %pagecolor=RoyalBlue,
    %urlcolor=Black, linkcolor=Black, citecolor=Black, %pagecolor=Black,
    pdftitle={\myTitle},
    pdfauthor={\textcopyright\ \myName, \myUni, \myFaculty},
    pdfsubject={},
    pdfkeywords={},
    pdfcreator={pdfLaTeX},
    pdfproducer={LaTeX}
}

%**************************************************************
% Impostazioni di itemize
%**************************************************************
\renewcommand{\labelitemi}{$\ast$}

%\renewcommand{\labelitemi}{$\bullet$}
%\renewcommand{\labelitemii}{$\cdot$}
%\renewcommand{\labelitemiii}{$\diamond$}
%\renewcommand{\labelitemiv}{$\ast$}


%**************************************************************
% Impostazioni di listings
%**************************************************************
\lstset{
    language=[LaTeX]Tex,%C++,
    keywordstyle=\color{RoyalBlue}, %\bfseries,
    basicstyle=\small\ttfamily,
    %identifierstyle=\color{NavyBlue},
    commentstyle=\color{Green}\ttfamily,
    stringstyle=\rmfamily,
    numbers=none, %left,%
    numberstyle=\scriptsize, %\tiny
    stepnumber=5,
    numbersep=8pt,
    showstringspaces=false,
    breaklines=true,
    frameround=ftff,
    frame=single
} 


%**************************************************************
% Impostazioni di xcolor
%**************************************************************
\definecolor{webgreen}{rgb}{0,.5,0}
\definecolor{webbrown}{rgb}{.6,0,0}


%**************************************************************
% Altro
%**************************************************************

\newcommand{\omissis}{[\dots\negthinspace]} % produce [...]

% eccezioni all'algoritmo di sillabazione
\hyphenation
{
    ma-cro-istru-zio-ne
    gi-ral-din
}

\newcommand{\sectionname}{sezione}
\addto\captionsitalian{\renewcommand{\figurename}{figura}
                       \renewcommand{\tablename}{tabella}}

\newcommand{\glsfirstoccur}{\ap{{[g]}}}

\newcommand{\intro}[1]{\emph{\textsf{#1}}}

%**************************************************************
% Environment per ``rischi''
%**************************************************************
\newcounter{riskcounter}                % define a counter
\setcounter{riskcounter}{0}             % set the counter to some initial value

%%%% Parameters
% #1: Title
\newenvironment{risk}[1]{
    \refstepcounter{riskcounter}        % increment counter
    \par \noindent                      % start new paragraph
    \textbf{\arabic{riskcounter}. #1}   % display the title before the 
                                        % content of the environment is displayed 
}{
    \par\medskip
}

\newcommand{\riskname}{Rischio}

\newcommand{\riskprob}[1]{\textbf{\\Probabilità:} #1.}

\newcommand{\riskimpact}[1]{\textbf{\\Impatto:} #1.}

\newcommand{\riskdescription}[1]{\textbf{\\Descrizione:} #1.}

\newcommand{\risksolution}[1]{\textbf{\\Gestione:} #1.}

%**************************************************************
% Environment per ``use case''
%**************************************************************
\newcounter{usecasecounter}             % define a counter
\setcounter{usecasecounter}{0}          % set the counter to some initial value

%%%% Parameters
% #1: ID
% #2: Nome
\newenvironment{usecase}[2]{
    \renewcommand{\theusecasecounter}{\usecasename #1}  % this is where the display of 
                                                        % the counter is overwritten/modified
    \refstepcounter{usecasecounter}             % increment counter
    \vspace{10pt}
    \par \noindent                              % start new paragraph
    {\large \textbf{\usecasename #1: #2}}       % display the title before the 
                                                % content of the environment is displayed 
    \medskip
}{
    \medskip
}

\newcommand{\usecasename}{UC}

\newcommand{\usecaseactors}[1]{\textbf{\\Attori Principali:} #1. \vspace{4pt}}
\newcommand{\usecasepre}[1]{\textbf{\\Precondizioni:} #1. \vspace{4pt}}
\newcommand{\usecasedesc}[1]{\textbf{\\Descrizione:} #1. \vspace{4pt}}
\newcommand{\usecasepost}[1]{\textbf{\\Postcondizioni:} #1. \vspace{4pt}}
\newcommand{\usecasealt}[1]{\textbf{\\Scenario Alternativo:} #1. \vspace{4pt}}

%**************************************************************
% Environment per ``namespace description''
%**************************************************************

\newenvironment{namespacedesc}{
    \vspace{10pt}
    \par \noindent                              % start new paragraph
    \begin{description} 
}{
    \end{description}
    \medskip
}

\newcommand{\classdesc}[2]{\item[\textbf{#1:}] #2}